Полное прохождение в виде команд доступно в файле autoscript.

Данное руководство пытается в более деликатной форме намекнуть на пути преодоления сложностей, возникающих при прохождении игры.

\subsection*{Разговор с Ларисой}

\reflectbox{Вы в любом случае пройдёте эту сцену.}

\reflectbox{Но имейте в виду, что конфликт можно "потушить".}

\subsection*{Торможение (выход на орбиту Луны)}

\reflectbox{Почти всё делается автоматически с помощью комьютера.}

\reflectbox{Просто разговаривайте с Сергеем, он будет вводить программы,}

\reflectbox{а вы -- запускать их на выполнение.}

\reflectbox{Здесь можно говорить с ЦУП. Иногда придётся просто ждать.}

\reflectbox{Кстати, "ждать" можно сокращать до "ж".}

\subsection*{Расстыковка}

\reflectbox{Вы должны научиться общаться по радио.}

\reflectbox{Запомните позывные: Заря, Арго, Беркут.}

\reflectbox{Вам придётся на протяжении всей игры поддерживать связь.}

\reflectbox{Стыковочные замки доступны после открытия люка. Осмотрите люк.}

\reflectbox{Правая ручка - это вращение корабля. Левая - его движение.}

\reflectbox{Модуль нужно отвести от Арго.}

\reflectbox{Для этого вы сначала двигаетесь назад}

\reflectbox{("двигать левую ручку назад", или "двигать левую назад"),}

\reflectbox{затем разворачиваетесь на 180 градусов (двигать правую назад).}

\reflectbox{Впрочем, можно и вперёд.}

\reflectbox{Главное, модуль будет поворачиваться по тангажу.}

\reflectbox{После этого, отлетите ещё дальше (двигать левую вперёд).}

\subsection*{Посадка}

\reflectbox{При посадке главное понять, что от вас требуется двигаться на восток.}

\reflectbox{А также тот факт, что при посадке двигатель модуля смотрит вниз, а нос -- вверх.}

\reflectbox{И то, что на орбите было кормой, стало низом.}

\reflectbox{Это означает, что правый рычаг теперь управляет тем, куда смотрят окна модуля.}

\reflectbox{Двигая правый рычаг вправо или влево поверните на восток.}

\reflectbox{Модуль по умолчанию выровнен, то-есть он просто садится медленно на Луну.}

\reflectbox{Чтобы придать ему горизонтальное движение, надо немного накренить его вперёд.}

\reflectbox{Двигать правую ручку вперёд.}

\reflectbox{То-есть, вы по тангажу опускаете нос модуля, который до этого смотрел строго вверх.}

\reflectbox{И тем самым, начинаете двигаться не только вниз, но и вперёд.}

\reflectbox{Вы можете прибавить скорость если ещё сильнее увеличите крен.}

\reflectbox{Дальше, просто ждите пока Александр не сообщит вам, что видимость в норме.}

\reflectbox{После чего выравнивайте модуль}

\reflectbox{(гасите горизонтальную скорость до 0, дёрнуть правую ручку назад).}

\reflectbox{Потом "ждать". После чего, если ваша горизонтальная скорость равна 0,}

\reflectbox{а вертикальная отрицательна и при этом видимость в порядке -- вы сядете.}

\reflectbox{Левый рычаг можно вообще не использовать,}

\reflectbox{с помощью него вы можете затормозить скорость падения}

\reflectbox{или наоборот, увеличить её.}

\subsection*{Луноход}

\reflectbox{Пеленгатор уже находится в луноходе.}

\reflectbox{Просто включите его и осматривайте}

\reflectbox{для получения информации о доступных направлениях.}

\subsection*{База}

\reflectbox{Контроллер можно получить из маяка (если его разобрать),}

\reflectbox{или сгонять за запасными запчастями к лунному модулю.}

\reflectbox{На базе вам нужно осмотреть всё, что получится}

\reflectbox{и потом выходить на связь с Зарёй.}

\reflectbox{Вам откроется новое направление - на Пик Малаперта.}

\reflectbox{Просто осмотрите пеленгатор.}

\subsection*{Пик Малаперта}

\reflectbox{Тут не должно возникнуть проблем.}

\subsection*{Во дворце с лунной принцессой}

\reflectbox{Тут тоже не должно возникнуть проблем.}

\subsection*{После встречи с лунной принцессой}

\reflectbox{Многие разрушенные предметы теперь содержат дополнительную}

\reflectbox{информацию о происшедшем на базе. Также, вам стоит осмотреть постер.}

\reflectbox{Вы можете не выполнять приказания принцессы буквально.}

\reflectbox{Например, можно не блокировать базу. Или блокировать, получить}

\reflectbox{новое задание и снова разблокировать.}

\reflectbox{Это принесёт некоторое моральное удовлетворение.}

\reflectbox{Так же не обязательно добивать космонавта.}

\reflectbox{Достаточно ударить его и забрать трубу.}

\reflectbox{Запчасти для трансмиттера придётся достать из панели управления лунного модуля.}

\subsection*{Встреча с Ларисой "во сне"}

\reflectbox{Чтобы выйти из порочного круга конфликта,}

\reflectbox{осмотрите Ларису в конце диалога.}

\reflectbox{Обратите внимание на слова, выделенные курсивом.}

\subsection*{Цветочная поляна}

\reflectbox{Проблем возникнуть не должно.}

\subsection*{Конец}

Спасибо за прохождение игры!

Надеюсь, вам было так же интересно, как и мне!
